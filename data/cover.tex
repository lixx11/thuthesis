\thusetup{
  %=========
  % 中文信息
  %=========
  ctitle={基于X射线自由电子激光的串行飞秒晶体学数据分析与实验研究},
  cdegree={工学博士},
  cdepartment={工程物理系},
  cmajor={核科学与技术},
  cauthor={李选选},
  csupervisor={张丽教授},
  ccosupervisor={刘海广特聘研究员}, % 联合指导老师
  %=========
  % 英文信息
  %=========
  etitle={Serial Femtosecond Crystallography at XFEL Facilities: Data Analysis and Experimental Study},
  edegree={Doctor of Philosophy},
  emajor={Nuclear Science and Technology},
  eauthor={Li Xuanxuan},
  esupervisor={Professor Zhang Li},
  eassosupervisor={Professor Liu Haiguang},
}

% 定义中英文摘要和关键字
\begin{cabstract}
  TODO
  % 论文的摘要是对论文研究内容和成果的高度概括。摘要应对论文所研究的问题及其研究目
  % 的进行描述,对研究方法和过程进行简单介绍,对研究成果和所得结论进行概括。摘要应
  % 具有独立性和自明性,其内容应包含与论文全文同等量的主要信息。使读者即使不阅读全
  % 文,通过摘要就能了解论文的总体内容和主要成果。

  % 论文摘要的书写应力求精确、简明。切忌写成对论文书写内容进行提要的形式,尤其要避
  % 免“第 1 章……;第 2 章……;……”这种或类似的陈述方式。

  % 本文介绍清华大学论文模板 \thuthesis{} 的使用方法。本模板符合学校的本科、硕士、
  % 博士论文格式要求。

  % 本文的创新点主要有:
  % \begin{itemize}
  %   \item 用例子来解释模板的使用方法;
  %   \item 用废话来填充无关紧要的部分;
  %   \item 一边学习摸索一边编写新代码。
  % \end{itemize}

  % 关键词是为了文献标引工作、用以表示全文主要内容信息的单词或术语。关键词不超过 5
  % 个,每个关键词中间用分号分隔。(模板作者注:关键词分隔符不用考虑,模板会自动处
  % 理。英文关键词同理。)
\end{cabstract}

\ckeywords{X射线自由电子激光, 串行飞秒晶体学, 预处理, 指标化, 泵浦探测}

\begin{eabstract}
  TODO
\end{eabstract}

\ekeywords{X-ray free-electron laser, serial femtosecond crystallography, preprocess, index, pump-probe}
