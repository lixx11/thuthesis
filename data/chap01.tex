\chapter{引言}

\section{X射线自由电子激光}
\subsection{X射线自由电子激光发展历史及现状}

X射线自由电子激光(XFEL, X-ray free-electron laser)的发展最早可以追溯到上世纪50年代。1951年美国斯坦福大学微波实验室的科学家Motz第一次提出波荡器的概念,即利用一系列极性交替的磁铁形成交变的磁场,当相对论性电子通过这种交变磁场时可以在毫米波尺度内产生相干辐射\cite{motz1951applications}。1953年Motz及其同事使用斯坦福线性加速器,利用3-5MeV电子束和平板波荡器,成功产生了毫米波的相干辐射。当电子能量提高到100MeV时,可以产生非相干的可见光\cite{motz1953experiments}。但是这种方法在向更短波长推广时遇到了很大的困难,如果要产生1Å(1Å=0.1nm)的自发辐射,需要将电子束压缩到~0.1Å的尺度,这在工程上很难实现。1960年Philips在Ubitron(一个基于波荡器的微波源)的工作表明电子在通过波荡器时可以发生自发的调制(bunching),自发辐射的能量来自于电子在轴向方向的动能\cite{phillips1988history}。

自由电子激光(FEL,free-electron laser)的概念是1971年由美国斯坦福大学的科学家Madey提出的。在FEL体系中,电磁辐射被纳入到电子-波荡器系统中,Madey从量子力学的角度,利用Weizsacker–Williams方法计算了电子束在通过波荡器时的小信号低增益过程,证明了FEL在远红外到可见光波长范围内的可行性,并提出实现紫外甚至X射线波段相干辐射的可能性\cite{madey1971stimulated}。Madey研究组通过实验进一步证明了FEL的可行性。1976年,Elias等人利用螺旋波荡器成功观测到10.6μm的信号放大(单程增益7\%)。1977年,Deacon等人实现了第一个FEL装置,该装置为低增益的振荡器型FEL,工作波长3.4μm,输入电子能量43MeV,光学谐振腔长12.7米\cite{deacon1977first}。在这种振荡器型FEL中,电磁辐射在谐振腔中多次往返,每次经过波荡器强度都有小幅增长,经过若干次放大,最终达到很高的能量输出。但是这种模式的FEL在短波长时增益下降,而且需要反射镜实现放大,导致很难进一步向X射线波段发展。

在Madey提出FEL概念并在实验上证实了其可行性之后,许多科学家提出了新的理论解释以及工作模式。Colson\cite{colson1977one}和Hopf\cite{hopf1976classical}用经典理论重新解释了Madey的低增益FEL系统。后续的一些研究工作提出了FEL的高增益模式。在高增益模式中,电子束在波荡器中只通过一次,初始的电磁辐射通过指数放大直至饱和。这种模式不需要光学谐振腔,而且可以使用自发辐射最为初始信号,即所谓的自放大自发辐射(SASE,Self-Amplified Spontaneous Emission)。1982年Derbenev等人基于高增益理论,提出可以利用1GeV电子的储存环(storage ring)实现软X射线激光的可行性\cite{derbenev1982possibility}。


XFEL的产生需要高质量的电子束。上世纪80年代光阴极微波电子枪等技术的发展使得高增益的短波长XFEL成为可能。1998年Hogan等人实现了第一个SASE模式的FEL,在红外波段下实现了600\%的增益\cite{hogan1998measurements1},并在几个月后实现了12μmFEL的$10^5$的增益\cite{hogan1998measurements2}。2003年德国DESY的Ischebeck研究组利用TESLA实验设施进行了100nmFEL的衍射实验,验证了SASE模式FEL的横向相干性\cite{ischebeck2003study}。2007年德国FLASH实现了13.7nm的SASE出光,脉冲长度10fs\cite{ackermann2007operation}。2009年是XFEL发展历史上的一个重要里程碑,世界上第一个硬X射线波段的自由电子激光设施LCLS建成并投入使用\cite{emma2010first}。LCLS最短可以产生1.2Å波长的X射线脉冲,脉冲长度可以在500fs到<10fs区间内调节。2011年日本的SACLA设施建成并出光,可以产生0.6Å波长的脉冲\cite{pile2011x}。到目前为止,世界上建成或在建的XFEL设施还包括意大利的FERMI FEL,欧洲的European XFEL,韩国的PAL-XFEL,瑞士的SwissFEL以及我国软X射线波段的SXFEL和硬X射线波段的SHINE。表x.x总结了各个XFEL设施的基本参数。

我国的自由电子激光的相关研究起始于上世纪80年代。1985年,中国科学院上海光学精密机械研究所实现了拉曼型FEL\cite{赵振堂47x}。1993年,由北京高能物理研究所建设的BFEL实现了红外波段的振荡型FEL,并于12月实现饱和出光,并达到了$10^8$的功率增益\cite{xie1995saturation}。2005年中国工程物理研究院远红外FEL实现出光\cite{li2005first}。2014年底,我国第一个XFEL装置SXFEL\cite{zhao2016seeded}在上海应用物理研究所张江园区动工,其最终目的是产生8.8nm波长的全相干软X射线脉冲,目前已经完成试验装置,用户装置正在建设中。另外,采用超导直线加速器的硬XFEL装置SHINE\cite{zhao2018sclf}也已于2018年开始动工建设,首期将建设三条波荡线,提供从软X射线到超硬X射线的飞秒级XFEL激光,脉冲频率高达100万Hz,与美国LCLS-II和欧洲的European XFEL相当。

\subsection{X射线自由电子激光的基本原理}
\subsubsection{电子在波荡器中的运动}

考虑一个平板波荡器,其磁感应强度在$y=0$处沿$z$方向以正弦形式变化,具体为:

\begin{equation}
\bm{B} = -B_{0}\text{sin}(k_{u}z) \bm{e}_{y}
\end{equation}

电子在磁场中受到洛伦兹力作用,

\begin{equation}
\gamma m_{e} \dot{\bm{v}} = - e \bm{v} \times \bm{B}
\end{equation}

应用$v_{z}=\dot{z} \approx v = \beta c = const$以及$v_{x} \ll v_{z}$,并代入初值条件,

$$x_{0} = 0, \quad \dot{x}(0) = \frac{eB_{0}}{\gamma m_{e} k_{u}}$$

可以得到电子在波荡器中的运动方程,
\begin{equation}
x(z) = \frac{K}{\beta \gamma k_{u}} \text{sin} (k_{u}z)
\end{equation}

从上述运动方程可以看出,电子在波荡器中以正弦形式运动,周期与波荡器周期相同。

\subsubsection{自发辐射}

在以电子纵向平均速度前进的运动坐标系中,利用洛伦兹变化得到电子在该坐标系内的运动。在忽略纵向的震荡情况下,相当于电子横向做简谐运动,其发出的电磁波的频率与电子简谐运动频率相同。然后变换回实验室坐标系中,得到实验室坐标系下的自发辐射的波长公式,

\begin{equation}
\lambda_{l} = \frac{\lambda_{u}}{2 \gamma ^{2}} (1 + \frac{K^{2}}{2} + \gamma^{2} \theta ^{2})
\end{equation}

其中$\theta$为相对于电子束前进方向$z$的发射角。

\subsubsection{共振条件}

假设波荡器中注入波长为$\lambda_{l}$的种子激光,其电场描述为

\begin{equation}
E_{x}(z,t) = E_{0}\text{cos}(k_{l}z - \omega_{l} t + \psi _{0})
\end{equation}

其中$k_{l} = \omega _{l} / c = 2\pi / \lambda_{l}$。电子受到该辐射场的作用,能量对时间的导数为

\begin{equation}
\frac{dW}{dt} = \bm{v \cdot F} = -e v_{x}(t)E_{x}(t)
\end{equation}

电子持续向辐射场提供能量的一个定性的条件即共振条件为:在电子走过半个波荡器周期$\lambda_{u}$时,电磁波比电子多行进半个波长$\lambda_{l}$。

共振条件用公式描述为

\begin{equation}
\frac{\lambda_{u} / 2}{\bar{v}_{z}} = \frac{\lambda_{u} / 2 + \lambda_{l} /2}{c}
\end{equation}

代入式,使用近似条件$\bar{v}_{z} \approx c$,可以得到符合共振条件的波长为

\begin{equation}
\lambda_{l} = \frac{\lambda_{u}}{2\gamma ^{2}}(1 + \frac{K^{2}}{2})
\end{equation}

自发辐射波长正好符合共振条件,这也就解释了为什么自发辐射可以作为XFEL的种子激光。

\subsubsection{电磁辐射的指数放大}

波荡器中对电子与电磁波之间的相互作用的一个定性的解释是:不同的电子相对电磁波具有不同的相位,部分电子从电磁波获取能量,部分电子损失能量,从而导致较快的电子减速,较慢的电子加速,最终实现电磁辐射对电子密度的调制(microbunching)。整体的效果是电子在波荡器中运动时将一部分动能转移给电磁辐射,实现电磁辐射的增益。根据单程增益的大小,FEL可以分为低增益和高增益两种模式。目前大部分的XFEL设施采用的SASE就是一种高增益模式,利用波荡器中高速电子的自发辐射与电子的相互作用,通过指数放大($I=I_{0}e^{z/L_{G0}}$),最终实现相干性极好的脉冲式X射线激光。

一维情况下,功率增益长度

\begin{equation}
L_{G0} = \frac{\lambda_{u}}{4\pi \sqrt{3}\rho}
\end{equation}

其中$\rho$为高增益FEL的Pierce参数,定义为

\begin{equation}
\rho 
= \left[\frac{K^{2}[JJ]^{2}}{32}\frac{k_{p}^{2}}{k_{u}^{2}}\right]^{1/3}
= \left[\frac{1}{16}\frac{I_{e}}{I_{A}}\frac{K^{2}[JJ]^{2}}{\gamma^{3}\sigma_{x}^{2}k_{u}^2}\right]^{1/3}
\end{equation}

\subsection{X射线自由电子激光的应用领域}

\section{利用X射线衍射方法解析晶体结构的原理}

\section{串行晶体学数据分析的流程与现状}

\section{论文的研究内容和结构安排}
